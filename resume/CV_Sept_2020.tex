\documentclass[11pt, letterpaper, sans]{moderncv}

\moderncvstyle{classic}
\moderncvcolor{blue}
\usepackage{multicol}
\usepackage{lipsum}
\usepackage{courier}
\usepackage[colorlinks = true,
            linkcolor = blue,
            urlcolor  = blue,
            citecolor = blue,
            anchorcolor = blue]{}
\usepackage[hyperref]{}
\usepackage[scale = 0.9]{geometry}

%------------------------------------------------------------
%       NAME AND CONTACT INFORMATION
%------------------------------------------------------------
\firstname{Gang}
\familyname{Yang}

\address{9450 Gilman Drive, La Jolla}{San Diego, California, US, 92093}
\mobile{(+1) 858 666 7190 (NOT Available NOW)}
\email{yanggangjack@gmail.com}

\begin{document}

\makecvtitle

% \textbf{Website}: \href{https://immmjack.github.io/}{https://immmjack.github.io/} \\*
% \textbf{GitHub}: \href{https://github.com/immmjack}{immmjack} \\*
% \textbf{LinkedIn}: \href{https://www.linkedin.com/in/gang-yang-603708195/}{Gang Yang} 


\section{Education}

\cventry{2019 --}{BS in Computer Science and Mathematics}{Jacobs School of Engineering}{University of California, San Diego}{GPA: 3.9/4.0, Expected 2023}{Courses: Introduction to Data Structures, Honors Linear Algebra, Foundations of Real Analysis, Introduction to Probability Theory, Stochastic Process, Mathematical Statistics}
\cventry{2016 -- 2019}{Ningbo Foreign Language School}{AP Department}{GPA: 4.56/4.0}{}{}{}

\section{Working Experience}
\cventry{Sept 2020 -- Dec 2020}{Computer Science Tutor at UCSD}{Jacobs School of Engineering}{}{}{Working under Professor Adalbert Gerald's CSE 8A: Introduction to Programming and Computational Problem Solving 1. My main task will be grading weekly programming assignments and helping students during lab hours.}

% \vspace*{0.2\baselineskip}
\cventry{Summer 2020}{Data Analyst at Donghai Marine Insurance Co., Ltd}{Information and Technology Department}{}{}{Designed and implemented a database system to visualize statistical data for each policy}

\section{Community}
\cventry{Blog}{\url{https://immmjack.github.io/}}{My personal website}{}{}{}
\cventry{GitHub}{\url{github.com/immmjack}}{}{}{}{}
\cventry{Kaggle}{\url{https://www.kaggle.com/immmjack}}{}{}{}{}
\cventry{LinkedIn}{\url{https://www.linkedin.com/in/gang-yang-603708195/}}{}{}{}{}

\section{Project Experience}

\cventry{September 2020}
{Survival Prediction of Titanic}
{Python}
{Kaggle Competition}{}
{
This is my first project in Machine Learning. The aim is to create a model that predicts which passengers survived the Titanic shipwreck. I used Decision Tree model with \textbf{k fold cross validation} with \texttt{cv = 10}. I will update my prediction soon since Decision Tree is not a great model to provide prediction.
}

% \vspace*{0.2\baselineskip}
\cventry{March 2020}
{Minesweeper}
{Java}
{Independent Project}{}
{
Minesweeper is a single-player puzzle game. The objective of the game is to clear a rectangular board containing hidden ``mines'' or bombs without detonating any of them, with help from clues about the number of neighboring mines in each field. I implemented the game via object orientation approach. The \texttt{Play} class starts the game as a driver. The \texttt{Bomb} class stores the information about game board. The \texttt{TimeChecker} class records the playing time. Others classes provide the implementation of User Interface.
}

\section{Awards}
\cventry{2019 --}{Provost Honors at John Muir College}{}{}{}{}
\cventry{April 2018}{Top 5\% Internationally in Euclid Mathematics Contest}{}{}{}{}

\section{Skills}
\cventry{Programming Languages}{Java, Python, R, TeX}{}{}{}{}
\cventry{Frameworks}{Matplotlib, Pandas, Sklearn, MySQL}{}{}{}{}
\cventry{Tools}{Git, Jupyter Notebook, R Markdown}{}{}{}{}
\cventry{Games}{Hearthstone (Legend Rank)}{Capricorn\#51956 (CN Server)}{}{}{}

\end{document}
